\documentclass{article}
\usepackage{geometry}
\usepackage{longtable}
\usepackage{array}
\usepackage{ragged2e}
\geometry{left=1in}
\newcolumntype{L}[1]{>{\raggedright\let\newline\\\arraybackslash\hspace{0pt}}p{#1}}
\begin{document}
\title {FINNGEN gene report for ((( gene ))) }
\author{FINNGEN}
\maketitle
\newpage
\newpage
((* set math = imp0rt('math') *))
\section{ Basic gene info }
\textbf{GENE}: ((( gene ))) \\
\medskip
\textbf{Location}: ((( geneinfo["maploc"] ))) BP: ((( geneinfo["start"] )))-((( geneinfo["stop"] ))) \\
\medskip
\textbf{Description:} ((( geneinfo["description"] ))) \\
\medskip
\textbf{SUMMARY:} \\
((( geneinfo["summary"] )))
\newpage



\begingroup
\renewcommand{\arraystretch}{1.5}
\section{ Functional variants }
\begin{longtable}{c|c|c|c|c|L{5cm} }
\caption{ Functional variant associations }\\
Variant & rsid & Consequence & MAF & N phenos p\textless((( "{:.0e}".format(func_var_assoc_threshold) ))) & top phenos \\
\hline
((* for var in functionalVars *))
    ((( var.variant | escape_tex))) & ((( var.rsid ))) & (((  var.consequence | escape_tex))) & ((( "{:.2e}".format(var.maf) ))) & ((( var.nSigPhenos ))) & (((var.sigPhenos| escape_tex)))  \\
((* endfor *))

\end{longtable}
\newpage

\section{ Top associations in the gene region }
\begin{longtable}{ >{\RaggedRight}p{3cm}|>{\RaggedRight}p{3cm}|c|c|c|c|c|c }
\caption{ Phenotype associations p\textless (((gene_top_assoc_threshold))) }\\
Variant & pheno & cases & controls & MAF case & MAF control & OR & p-value  \\
\hline
((* for var in topAssoc *))
((( var.id.replace(":"," ") | escape_tex))) \newline ((( var.rsids ))) & ((( var.phenostring | escape_tex))) & (((var.num_cases ))) & (((var.num_controls))) & ((( "{:.2e}".format(var.maf_case) ))) & ((( "{:.2e}".format(var.maf_control) ))) & ((( "{:.2f}".format(math.exp(var.beta)) ))) & ((( "{:.2e}".format(var.pval) ))) \\
((* endfor *))
\end{longtable}



\newpage

\section{ Known GWAS hits in the region}
\begin{longtable}{ >{\RaggedRight}p{3cm}|L{3cm}|c|c|c|c|>{\RaggedRight}p{3cm} }
\caption{ Known GWAS hits in the region }\\
Variant & pheno & effective allele & EAF & beta/or & -10 log p-value & reference \\
\hline
((* for var in knownhits *))
((( var.variant.replace(":"," ") | escape_tex))) \newline ((( var.rsid ))) & ((( var.trait | escape_tex))) & ((( var.risk_allele ))) & ((( var.risk_frq ))) & ((( var.or_beta ))) & ((( var.log_pvalue ))) & (((var.study))) \\
((* endfor *))
\end{longtable}
\endgroup

\end{document}
